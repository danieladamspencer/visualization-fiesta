\documentclass{article}

\usepackage{amsmath}

\title{Coinflipping for Profit}

\begin{document}
	\maketitle
	
	Consider the following game with two players, Alice and Bob. Behind a curtain, Alice and Bob simultaneously flip biased coins (with probabilities $p_A$ and $p_B$ that they get heads, respectively) until one of them gets heads, and then they turn on a  light that you, the observer, can see. You wouldn't know who got the heads. The light itself comes on after $N$ flips, so what is the probability that Alice got heads? What about the probability that Bob got heads?
	
	First, let's think about the distributions for the number of flips that Bob or Alice might make, and call these $X_A$ and $X_B$. These values follow a negative binomial distribution ($X_i \sim \text{Negative Binomial}(r = 1, p = p_i); \quad i = (A,B)$). Thus, the random variable $N$, the number of flips needed to turn on the light can be thought of as
	
	$$ N = \min\left(X_A, X_B\right). $$
	
	The distribution of $N$ can be found using the definition of the minimum, that is
	
	\begin{align*}
		F_N(n) & = P(N \leq n) \\
		& = P(\min(X_A,X_B) \leq n) \\
		& = 1 - P(\min(X_A,X_B) > n).
	\end{align*}

	Now it's important to think about how the coin flips themselves are independent, so
	
	\begin{align*}
		P(\min(X_A,X_B) > n) & = P(X_A > n) P(X_B > n)
	\end{align*}

	and 
	
	\begin{align*}
		P(X_A > n) & = \sum_{x = n+1}^\infty \binom{x}{x - 1} (1 - p_A)^{x - 1} p_A \\
		& =  \sum_{x = n+1}^\infty x (1 - p_A)^{x - 1} p_A
	\end{align*}
	
\end{document}